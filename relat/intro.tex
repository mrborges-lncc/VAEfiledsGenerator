%%%%%%%%%%%%%%%%%%%%%%%%%%%%%%%%%%%%%%%%%%%%%%%%%%%%%%%%%%%%%%%%%%%%%%%%%%%%%%%%%%%%%%%%%
%%%%%%%%%%%%%%%%%%%%%%%%%%%%%%%%%%%%%%%%%%%%%%%%%%%%%%%%%%%%%%%%%%%%%%%%%%%%%%%%%%%%%%%%%
\section{Introdution}
%%%%%%%%%%%%%%%%%%%%%%%%%%%%%%%%%%%%%%%%%%%%%%%%%%%%%%%%%%%%%%%%%%%%%%%%%%%%%%%%%%%%%%%%%
Natural reservoirs display a high degree of variability in their hydraulic properties in multiple length scales \citep{dagan89, gelhar93}.
Such variability plays a strong impact on flu-id flow patterns in subsurface formations \citep{glimm92}.
Due to operational difficulties and high cost, direct measurements of the properties of interest are scarce and, therefore, a deterministic description of the reservoirs cannot be achieved.
Alternatively, a stochastic approach should be adopted to characterize uncertainties in reservoir parameters that, according to \cite{efendiev2006},  are the primary factor contributing to reservoir forecasting uncertainties.
In other words, the predictability of computational models is severely limited by the lack of an adequate description of reservoir properties.

Data acquisition along with the use of high-performance computing has encouragedThe growing data acquisition along with the use of high-performance computing has encouraged the use of dynamic data directly in simulations to reduce the associated uncertainties and increase the predictability of the models.
Well tests, production history, pressure variation on monitors, and tracer tests, among others, are direct measures of reservoir responses and can provide important information about the flow processes.

Matching field dynamic data with reservoir simulation results is a stochastic inverse problem that can be formalized in terms of Bayesian inference and Markov chain Monte Carlo methods (\mcmc).
The Bayesian inference is convenient in quantifying the added value of information from several sources, while \mcmc\ methods provide a computational framework for sampling from \apost\ distribution \citep{casella05, liu2008}.

Despite its high computational cost that, in some cases, can be prohibitive, \mcmc\ methods are regarded as the gold standard technique for Bayesian inference \citep{Nemeth2021}.
In 2000, Computing in Science \& Engineering, a joint publication of the IEEE Computer Society and the American Institute of Physics, placed the Metropolis Algorithm on the list of the 10 algorithms with the greatest impact on the development of science and engineering practice in the 20th century (“ Top Ten Algorithms of the Century”  \citep{top10}).

Within this context \mcmc\ methods have been
widely used in the last decades in porous media flow problems
\citep{liu2003,efendiev05,efendiev2006,ma2008,dostert10,das10,mondal10,ginting11,ginting12,iglesias13,emerick2013}.
%%%%%%%%%%%%%%%%%%%%%%%%%%%%%%%%%%%%%%%%%%%%%%%%%%%%%%%%%%%%%%%%%%%%%%%%%%%%%%%%%%%%%%%%%
\textcolor{red}{
\mcmc\ algorithms typically require the design of proposal mechanisms to generate candidate hypotheses.
Many existing machine learning algorithms can be adapted to become proposal mechanisms \citep{n2013variational}.
This is often essential to obtain \mcmc\ algorithms that converge quickly.}


%%%%%%%%%%%%%%%%%%%%%%%%%%%%%%%%%%%%%%%%%%%%%%%%%%%%%%%%%%%%%%%%%%%%%%%%%%%%%%%%%%%%%%%%%
%%%%%%%%%%%%%%%%%%%%%%%%%%%%%%%%%%%%%%%%%%%%%%%%%%%%%%%%%%%%%%%%%%%%%%%%%%%%%%%%%%%%%%%%%
\subsection{Random permeability fields}\label{fieldgeneratio}
%%%%%%%%%%%%%%%%%%%%%%%%%%%%%%%%%%%%%%%%%%%%%%%%%%%%%%%%%%%%%%%%%%%%%%%%%%%%%%%%%%%%%%%%%

Due to incomplete knowledge about the rock properties that show variability at multiple length scales, input parameters such as the permeability field, $\perm(\vx,\ale)$, are treated as random space functions with statistics inferred from geostatistical models (here $\vx =\left( x_{1},x_{2},x_{3}  \right)^{\! ^{\mathsf{T}}}\in \mathbb{R}^{3}$ and $\ale$ is a random element in the probability space).
In line with \cite{dagan89} and \cite{gelhar93} the permeability field is modeled as a log-normally distributed random space function
%
\begin{equation}
  \perm\fxw = \beta\exp\Big[\rho\Y\fxw\Big],
\label{field}
\end{equation}
%
\noindent where $\beta,\rho\in\mathbb{R}^{+}$ and $\Y\fxw \sim \normalf{\mev{\Y}}{\covn{\Y}}$ is a Gaussian random field
characterized by its mean $\mev{\Y} = \med{\Y\fx}$ and two-point covariance function

\begin{equation}
 \covv{\Y} = \covar{\Y\fx}{\Y\fy} = \mathsf{E}\Big[ \big( \Y\fx - \med{\Y\fx}  \big) \big( \Y\fy - \med{\Y\fy}  \big) \Big].
 \label{covg}
\end{equation}
%
\textcolor{black}{Moreover, in this work, $\Y$ is a second-order stationary process\index{second-order stationary process} \citep{gelhar93}, that is:}

\begin{equation}
 \begin{array}{rcl}
    \med{\Y\fx} &=& \me{\Y},\quad \mbox{(constant)}\\ \\
    \covv{\Y}   &=& \covn{\Y}\left(\lVert \vx - \vy \rVert \right) = \covn{\Y}\left(d\right).
 \end{array}
\end{equation}



%%%%%%%%%%%%%%%%%%%%%%%%%%%%%%%%%%%%%%%%%%%%%%%%%%%%%%%%%%%%%%%%%%%%%%%%%%%%%%%%%%%%%%%%%
%%%%%%%%%%%%%%%%%%%%%%%%%%%%%%%%%%%%%%%%%%%%%%%%%%%%%%%%%%%%%%%%%%%%%%%%%%%%%%%%%%%%%%%%%
\subsection{\KL\ expansion}
%%%%%%%%%%%%%%%%%%%%%%%%%%%%%%%%%%%%%%%%%%%%%%%%%%%%%%%%%%%%%%%%%%%%%%%%%%%%%%%%%%%%%%%%%

The Gaussian field $\Y$ can be represented as a series expansion involving a complete set of deterministic functions with correspondent random coefficients using the \KL\ (\kl) expansion proposed independently by \cite{karhunen46} and \cite{loeve55}.
It is based on the eigen-decomposition of the covariance function.
Depending on how fast the eigenvalues decay one may be able to  retain only a small number of terms in a truncated expansion and, consequently, this procedure may reduce the search to a smaller parameter space.
In uncertainty quantification methods for porous media flows, the \kl\ expansion has been widely used to reduce the number of parameters used to represent the permeability field \citep{efendiev05,efendiev2006,das10,mondal10,ginting11,ginting12}.
Another advantage of \kl\ expansion lies on the fact that it provides orthogonal deterministic basis functions and uncorrelated random coefficients, allowing for the optimal encapsulation of the information contained in the random process into a set of discrete uncorrelated random variables \citep{GhanemSpanos}.
This remarkable feature can be used to simplify the Metropolis-Hastings \mcmc\ Algorithm in the sense of the search may be performed in the space of discrete uncorrelated random variables ($\vtheta$), no longer in the space of permeabilities which have a more complex statistical structure.

In summary, we may have the reduction of the stochastic dimension and the sampling in the space of uncorrelated Gaussian variables.
